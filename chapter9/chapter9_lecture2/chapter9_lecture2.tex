% Lecture Template for ENGR1120-800 -Tristan Hill - Spring 2020
% Honors Programmming for Engineers
% Time Response - Lecture 3

% I am finally converting my stuff to BEAMER

% Document settings

%\documentclass{beamer}                  % for presentation ?
\documentclass[handout]{beamer}  % for handout ?
\usepackage{beamerthemesplit}
\usepackage{amsmath}
\usepackage{listings}
\usepackage{multicol}
\usepackage{framed}

\beamertemplateballitem

\definecolor{TTUpurple}{rgb}{0.3098, 0.1607, 0.5176} % TTU Purple (primary)
\definecolor{TTUgold}{rgb}{1.0000, 0.8666, 0.0000} % TTU Gold (primary)

\setbeamercolor{palette primary}{bg=TTUpurple,fg=TTUgold}
\setbeamercolor{palette secondary}{bg=black,fg=TTUgold}
\setbeamercolor{palette tertiary}{bg=black,fg=TTUpurple}
\setbeamercolor{palette quaternary}{bg=TTUgold,fg=black}
\setbeamercolor{structure}{fg=TTUpurple} % itemize, enumerate, etc
\setbeamercolor{section in toc}{fg=TTUpurple} % TOC sections

\hypersetup{
%    bookmarks=true,         % show bookmarks bar?
%    unicode=false,          % non-Latin characters in Acrobat’s bookmarks
%    pdftoolbar=true,        % show Acrobat’s toolbar?
%    pdfmenubar=true,        % show Acrobat’s menu?
%    pdffitwindow=false,     % window fit to page when opened
%    pdfstartview={FitH},    % fits the width of the page to the window
%    pdftitle={My title},    % title
%    pdfauthor={Author},     % author
%    pdfsubject={Subject},   % subject of the document
%    pdfcreator={Creator},   % creator of the document
%    pdfproducer={Producer}, % producer of the document
%    pdfkeywords={keyword1} {key2} {key3}, % list of keywords
%    pdfnewwindow=true,      % links in new window
   % colorlinks=true,       % false: boxed links; true: colored links
 %   linkcolor=red,          % color of internal links (change box color with linkbordercolor)
  %  citecolor=green,        % color of links to bibliography
   % filecolor=magenta,      % color of file links
  %  urlcolor=blue           % color of external links
}

%\usefonttheme{professionalfonts}
\definecolor{mygray}{rgb}{.6, .6, .6}

% [153,50,204] - dark orchid
\definecolor{mypurple}{rgb}{0.6,0.1961,0.8}
%[139,69,19] - saddle brown
\definecolor{mybrown}{rgb}{0.5451,0.2706,0.0745}
\definecolor{mygreen}{rgb}{0,.4,0}
\definecolor{mygray}{rgb}{.6, .6, .6}
\definecolor{mypurple}{rgb}{0.6,0.1961,0.8}
\definecolor{mybrown}{rgb}{0.5451,0.2706,0.0745}
\definecolor{mygreen}{rgb}{0, .39, 0}

\newcommand{\R}{\color{red}}
\newcommand{\B}{\color{blue}}
\newcommand{\BR}{\color{mybrown}}
\newcommand{\K}{\color{black}}
\newcommand{\G}{\color{mygreen}}
\newcommand{\PR}{\color{mypurple}}

\newcommand{\vspccc}{\vspace{6mm}\\} % large vertical space
\newcommand{\vspcc}{\vspace{4mm}\\}   % medium vertical space
\newcommand{\vspc}{\vspace{2mm}\\}     % small vertical space

\newcommand{\hspccc}{\hspace{6mm}} % large horizontal space
\newcommand{\hspcc}{\hspace{4mm}}   % medium horizontal space
\newcommand{\hspc}{\hspace{2mm}}     % small horizontal space

\newcommand{\CNUM}{9\hspace{2mm}} % Chapter Number 
\newcommand{\LNUM}{2\hspace{2mm}} % Lecture Number 

\newcommand{\Lagr}{\mathcal{L}} % lagrangian

\newcommand{\secondtitle}{Data Input from .csv Files}% second line of the title of this presentation , aka the topic of this lecture

\title{Chapter 9 - Lecture \LNUM}
\author{ENGR1120 - 800 - Honors Programming for Engineers} % original formatting from Mike Renfro, September 21, 2004

\date{April 21, 2020}

\begin{document}

\lstset{language=MATLAB,basicstyle=\ttfamily\large,showstringspaces=false}

% Title page1 
\frame{\titlepage \center\textbf{\secondtitle}\vspcc}


% Section 0: Outline
\frame{

\large \textbf{Lecture \LNUM - \secondtitle} \vspc

 \begin{itemize}

	
	\item Review Data Files and  IO\vspc  % Section 1

	\item File Output to .csv  \vspc % Section 3

	\item Example in MATLAB \vspc % Section 4

\end{itemize}

}


%Section 1: What is a Data File? 
\section{Data Files}

\frame{
\subsection{What is a Data File?}
\frametitle{What is a Data File?}


	\textbf{ \Large What is a \PR Data  File \K? }\vspc
	
	\begin{itemize}
	\item  \large Standard way of organizing data for computer {\bf storage} \vspc
	
	\item  \large The data can represent many different things but it is all stored \G digitally\K\vspc
	
	\item \large Different \color{mypurple}file types \color{black} are used for different purposes \\
		\begin{itemize}
		\item \vspace{2mm}
		\item \vspace{2mm}
		\item \href{https://en.wikipedia.org/wiki/List_of_file_formats}{list of all...}\vspace{2mm}
		\end{itemize}
	\end{itemize}
}

\frame{
\subsection{Why use Data  Files? }
\frametitle{Why use Data  Files? }


         \Large Why use \PR Data  Files \K? \vspc
	\large
	\begin{itemize}
		\item Organize large amounts of information\vspc
		\item Share large amounts of information\vspc
		\item 
	\end{itemize}

}

%Section 2: File IO
\section{File IO}

\subsection{What is File IO?}
\frame{
\frametitle{What is File IO?}

\large What is File IO?
\begin{itemize}

	\item \vspace{5mm} 	
	\item \vspace{5mm}
	\item \vspace{5mm}

\end{itemize}


}

\subsection{File Output to a Program}
\frame{
\frametitle{File Output to a Program}

{\Large File Output to a Program}  \vspc

	\begin{itemize}
	\item  put data  into a file during \color{mypurple}Program Execution \color{black}  \vspc
	\item data comes from a variable(s) in your program  \vspc
	\end{itemize}

}

%Section 3: .csv Files
\section{.csv Files}

\subsection{Comma Separated Values}

\frame{
\frametitle{Comma Separated Values}



\large The individual values in a file are often separated or {\it delimited} by a comma. Other characters are also used such as the space or {\it newline}. \vspc
A .txt file with values delimited by commas is called a comma separated value file or {\bf .csv} file. \vspccc

\small
Traditionally the {\it end of file} was marked by a special character as well but modern data files are organized by file size and do not require an end of file character. \vspccc

}





%Section 4: File Output to MATLAB
\section{File Input in MATLAB}

\subsection{MATLAB functions for file IO}
\frame[containsverbatim]{
\frametitle{MATLAB functions for file IO}

{\large There are different ways to get data from a file. We are going to {\bf scan} the data one character at a time using these functions.}

\begin{framed}
\begin{lstlisting}
fopen()  

fprintf()

fclose()
\end{lstlisting}
\end{framed}

	

}

\subsection{The fopen() Function}
\frame[containsverbatim]{
\frametitle{The fopen() Function}

{\large Open the file with the {\bf fopen()} function\vspc}

\begin{framed}
\begin{lstlisting}
[FID]=fopen(FILENAME,PERMISSION)
\end{lstlisting}
\end{framed}

	\begin{itemize}

		\item \underline{Input 1:} FILENAME - the name of the file to open \vspc
		\item \underline{Input 2:} PERMISSION - direction  of access 'r' or 'w'  \vspc
		\item \underline{Output 1:} FID -  the file identifier \vspc

	\end{itemize}

}

\frame{
\frametitle{The File Identifier}

{\Large The {\bf file identifier} (FID) gives important info}\vspc
\large
	\begin{itemize}
 		\item If the file opens properly the FID will have a positive value\vspace{2mm} \vspc
	
		\item The FID will have a negative value if there was an \R error \K  \vspc		\begin{itemize}
			\item File is not in the proper \color{blue} directory \color{black}  \vspc
			\item The \color{mypurple} current folder  \color{black} has not been set properly\vspc

			\item Please organize you file structure! \vspc
		
		\end{itemize}

		\item \large  FID can also give information about the \color{blue} End Of File\color{black} \vspc
	
	% \scalebox{1.3}{{\fontfamily{qcr}\selectfont feof(fid) \color{mygreen} \% returns a boolean }} \vspc
 	
 	%\textbf{ \large  \color{black} We use this is we do not know how long the file is}\vspc
	
	% \scalebox{1.3}{{\fontfamily{qcr}\selectfont \color{blue}while\color{black}($\sim$feof(fid)) \color{mygreen}  }} \vspc
	\end{itemize}

}

\subsection{The fprintf() Function}

\frame[containsverbatim]{
\frametitle{The fprintf() Function}

{\Large {\bf fprintf()} can access the data only if the file is open}

\begin{framed}
\begin{lstlisting}
fprintf(FID,FORMAT,VAR);
\end{lstlisting}
\end{framed}

\large
	\begin{itemize}

		\item \underline{Input 1:} FID - the {\bf file identifier} fid \vspc
		 \underline{Input 2:} FORMAT - format specification of the scan \vspc
		 \underline{Input 3:} VAR - variable to be put into FORMAT\vspc

	\end{itemize}

}

\subsection{The fclose() Function}
\frame[containsverbatim]{
\frametitle{The fclose() Function}

{\Large Remember to close the file with {\bf fclose()} } \vspc


\begin{framed}
\begin{lstlisting}
[ST]=fclose(FID)
\end{lstlisting}
\end{framed}

\large
\begin{itemize}

		\item \underline{Input 1:} FID - the {\bf file identifier} fid \vspc
		\item \underline{Output 1:} ST -  status of close? \vspc

	\end{itemize}
\large  
	\begin{itemize}

 		\item Close the file after your program accesses the data  \vspc
		
		\item THIS IS EASY TO FORGET BUT IMPORTANT!!! \vspc

	\end{itemize}

}

\subsection{A Simple Example}

\frame[containsverbatim]{
\frametitle{A Simple Example}

\begin{framed}
\begin{lstlisting}
FID = fopen('output_data.csv','w');

x=999.9

A = fprintf(FID,'%f',x)

fclose(FID);
\end{lstlisting}
\end{framed}
}

\subsection{A More Complex Example}

\frame[containsverbatim]{
\frametitle{A More Complex Example}


\begin{framed}
\begin{lstlisting}
clear variables;close all;clc

FID = fopen('output_data.csv','w');

t=0:0.1:2*pi % some data to test
y=5*sin(3*t)

figure(1);plot(t,y)
\end{lstlisting}
\end{framed}

}

\subsection{A More Complex Example Continued}

\frame[containsverbatim]{
\frametitle{A More Complex Example Continued}


\begin{framed}
\begin{lstlisting}
j=1
while j<10
    fprintf(FID,'%f,',y(j));
    j=j+1;
end
      
fclose(FID);
\end{lstlisting}
\end{framed}
}

% references is not a section for now, for looks and it would be a waste of space
\frame{

\frametitle{References}

\begin{itemize}
	\item Your MATLAB textbook - Chapter 9 - Low Level File IO
\end{itemize}

}
\end{document}









 