% 
% ENGR 1120 030/031/E01  
% Tristan Hill -  Spring 2017 - Fall 2017 - Spring 2018
% Lab 3  - Basic User Input and Output 
%              - fprintf and input 
% 
 

% Document settings
\documentclass[11pt]{article}
\usepackage[margin=1in]{geometry}
\usepackage[pdftex]{graphicx}
\usepackage{multirow}
\usepackage{setspace}
\usepackage{hyperref}
\usepackage{color,soul}
\usepackage{fancyvrb}
\usepackage{framed}
\usepackage{wasysym}
\usepackage{multicol}

\pagestyle{plain}
\setlength\parindent{0pt}
\hypersetup{
    bookmarks=true,         % show bookmarks bar?
    unicode=false,          % non-Latin characters in Acrobat’s bookmarks
    pdftoolbar=true,        % show Acrobat’s toolbar?
    pdfmenubar=true,        % show Acrobat’s menu?
    pdffitwindow=false,     % window fit to page when opened
    pdfstartview={FitH},    % fits the width of the page to the window
    pdftitle={My title},    % title
    pdfauthor={Author},     % author
    pdfsubject={Subject},   % subject of the document
    pdfcreator={Creator},   % creator of the document
    pdfproducer={Producer}, % producer of the document
    pdfkeywords={keyword1} {key2} {key3}, % list of keywords
    pdfnewwindow=true,      % links in new window
    colorlinks=true,       % false: boxed links; true: colored links
    linkcolor=red,          % color of internal links (change box color with linkbordercolor)
    citecolor=green,        % color of links to bibliography
    filecolor=magenta,      % color of file links
    urlcolor=blue           % color of external links
}

% assignment number 
\newcommand{\NUM}{3} 
\newcommand{\VSpaceSize}{2mm} 
\newcommand{\HSpaceSize}{2mm} 

\newcommand{\secNum}{GSET: Programming}
\newcommand{\assnType}{Lab}
\newcommand{\assnTitle}{Basic Input and Output}
\newcommand{\assnNum}{3} 
\newcommand{\currTerm}{Summer 2022}

\definecolor{mygray}{rgb}{.6, .6, .6}

\setulcolor{red} 
\setstcolor{green} 
\sethlcolor{mygray} 

\begin{document}

	\textbf{\LARGE \secNum \hspace{1mm} - \hspace{1mm} \currTerm} \\\\
  \textbf{\LARGE \assnType \hspace{1mm}  \assnNum : \assnTitle} \\\\
	\textbf{\LARGE Simple Unit Conversions: Temperature} \\
	
	
	\begin{description}
        \vspace{3mm}
		\item [\textbf{ \Large Overview}] \textbf{ \Large :}\\
			\Large
			You will learn to program {\it user input} and {\it command window output} in MATLAB, using the {\it input} function and the {\it fprintf} function. You will complete a basic, but useful engineering calculation. The inputs are typed into the command window and the program will output the results to the command window. Also, the {\it character} and {\it string} data types will be introduced. \\
			
 	\item [\textbf{ \Large the {\it input()} function}] \textbf{ \Large :}\\   
            	\Large
            	\begin{itemize}
            		\item adds a simple user interface to your program
            		\item you type {\it command window input} 
            		\item the input can be stored as a variable	
            	\end{itemize}
        \vspace{5mm}
            
        \item [\textbf{ \Large the {\it fprintf()} function}] \textbf{ \Large :}\\   
            	\Large
            	\begin{itemize}
            		\item complete control of the {\it command window output}	
            		\item results in a formatted {\it string}, character level control 
            	\end{itemize}
            \vspace{5mm}
 \item [\textbf{ \Large Temperature Conversions}] \textbf{ \Large :}\\ \\
            The following equations describe conversions from degrees Fahrenheit (F) to Celsius (C) and Celsius (C) to Kelvin (K).\\

             \scalebox{1.3}{$^\circ C=(^\circ F-32) \times \frac{5}{9}$}   \hspace{15mm}    \scalebox{1.5}{$^\circ K=^\circ C +273.15$} \\ 
                                 
             \newpage                   
         


   
        \item [\textbf{ \Large Assignment}] \textbf{ \Large :}
            Write a MATLAB program (a script) to  do the following. Part 2 goes below part 1 {\it in the same script}:\\
            \begin{enumerate}
            \item Convert from F to C and K
            \begin{enumerate}
            \item
           With the {\it input} function, ask the user to enter a temperature value in units of {\it Fahrenheit}. Do not show the {\it default output}.
           
           \item 
           Convert the value to units of {\it Celsius}. Do not show the {\it default output}.
           
           \item 
           Convert the value to units of {\it Kelvin} . Do not show the {\it default output}.
           
           \item 
           With the {\it fprintf} function, show all three values with the proper units. Include 2 decimal places. \\
           
           
             \end{enumerate}
               \item Convert from K to C and F
            \begin{enumerate}
            \item
           With the {\it input} function, ask the user to enter a temperature value in units of {\it Kelvin}. Do not show the {\it default output}.
           
           \item 
           Convert the value to units of {\it Celsius}. Do not show the {\it default output}.
           
           \item 
           Convert the value to units of {\it Fahrenheit} . Do not show the {\it default output}.
           
           \item 
           With the {\it fprintf} function, show all three values with the proper units. Include 2 decimal places.
           
           
             \end{enumerate}
        	\end{enumerate}

\item [\textbf{Submission }]\textbf{:} \\
			\begin{itemize}
				\item Your program needs a proper {\it Header} or title block on it. Please see this discussion in the notes for details.\\
				\item Your script file needs to be named properly. Please see the {\it naming convention} document on ilearn. \\
				%\item Submit your file on ilearn in the {\it Laboratory \NUM} Assignments Folder. You can resubmit as many times as you would like but please wait at least 2 minutes between submissions. Your latest submission will be the only one graded.

			\end{itemize}


	\end{description}
 
\end{document}



