% Lecture Template for ENGR 1120 030 031 E01- Tristan Hill - Spring 2017 - Fall 2017
% 
% Introduction to MATLAB 
%
% THe Command WIndow, more IO
% input and fprintf

% Document settings
\documentclass[11pt]{article}
\usepackage[margin=1in]{geometry}
\usepackage[pdftex]{graphicx}
\usepackage{multirow}
\usepackage{setspace}
\usepackage{hyperref}
\usepackage{color,soul}
\usepackage{fancyvrb}
\usepackage{framed}
\usepackage{wasysym}
\usepackage{multicol}

\pagestyle{plain}
\setlength\parindent{0pt}
\hypersetup{
    bookmarks=true,         % show bookmarks bar?
    unicode=false,          % non-Latin characters in Acrobat’s bookmarks
    pdftoolbar=true,        % show Acrobat’s toolbar?
    pdfmenubar=true,        % show Acrobat’s menu?
    pdffitwindow=false,     % window fit to page when opened
    pdfstartview={FitH},    % fits the width of the page to the window
    pdftitle={My title},    % title
    pdfauthor={Author},     % author
    pdfsubject={Subject},   % subject of the document
    pdfcreator={Creator},   % creator of the document
    pdfproducer={Producer}, % producer of the document
    pdfkeywords={keyword1} {key2} {key3}, % list of keywords
    pdfnewwindow=true,      % links in new window
    colorlinks=true,       % false: boxed links; true: colored links
    linkcolor=red,          % color of internal links (change box color with linkbordercolor)
    citecolor=green,        % color of links to bibliography
    filecolor=magenta,      % color of file links
    urlcolor=blue           % color of external links
}

% assignment number 
\newcommand{\NUM}{3 } 
\newcommand{\VSpaceSize}{2mm} 
\newcommand{\HSpaceSize}{2mm} 

\definecolor{mygray}{rgb}{.6, .6, .6}

% [153,50,204] - dark orchid
\definecolor{mypurple}{rgb}{0.6,0.1961,0.8}
%[139,69,19] - saddle brown
\definecolor{mybrown}{rgb}{0.5451,0.2706,0.0745}

\setulcolor{red} 
\setstcolor{green} 
\sethlcolor{mygray} 

\begin{document}

\textbf{ \LARGE ENGR 1120 Lecture Chapter \NUM - \\\\ Command Window Input and Output} \\

\begin{itemize}

	\item \textbf{ \LARGE basic command window output ;}\\
		\Large
		\begin{itemize}
			\item by default each line of code (expression or assignment) will print to the C.W. \\ \vspace{20mm}
			\item however the {\it semicolon} is used to {\it suppress} the output\\ 
			\item {\bf example:} \\
			
				\scalebox{1.5}{{\fontfamily{qcr}\selectfont  >>x=10;}} \\\\
				
			\item the default output can be configured using the format command\\
		
				 \scalebox{1.5}{{\fontfamily{qcr}\selectfont  >>format compact}} \\\\
				 \scalebox{1.5}{{\fontfamily{qcr}\selectfont  >>format long}} \\\\
				 \scalebox{1.5}{{\fontfamily{qcr}\selectfont  >>format loose}} \\\\
				 \scalebox{1.5}{{\fontfamily{qcr}\selectfont  >>format short}} \\\\
					
			note: I will not be using the {\it format} command, you can look into if you wish\\		
		\end{itemize}
		
		\newpage
	\item \textbf{ \LARGE 2 new useful functions}\\
	
		
		\begin{itemize}
		
		       \item \textbf{ \Large the \LARGE{\it fprintf()} \Large function} \\
            	\Large
            	\begin{itemize}
            		\item similar to the C library function \\
            		\item complete control of the {\it command window output}	\\
            		\item also used with {\it file output}, {\bf f}ormatted {\bf print} to {\bf f}ile 	\\
            		\item results in a formatted {\it string}, character level control \\
            		
            		\item {\bf example:} \\\\
            						
            						The lines below goes in your code.  \vspace{10mm} \\			
            						\scalebox{1.5}{{\fontfamily{qcr}\selectfont my\textunderscore var=97.563;}} \\\\	
            						\scalebox{1.5}{{\fontfamily{qcr}\selectfont fprintf(\color{mypurple}`The value is \%f '\color{black},my\textunderscore var) }} \vspace{5mm} \\	
            						
            						The following output will appear in the command window.  \vspace{10mm} \\	
            						
            						\scalebox{1.5}{{\fontfamily{qcr}\selectfont >>The value is 97.563  }} \vspace{5mm} \\
            				
            						Can you tell what happened?  \vspace{10mm} \\	
            						
            	\end{itemize}
            	\newpage
            	\item \textbf{ \Large more about the \LARGE{\it fprintf()} \Large function} \vspace{10mm} \\
            	\scalebox{1.5}{{\fontfamily{qcr}\selectfont \color{mypurple}`\color{red}\% \color{blue}{\it feild width} \color{black}. \color{green}{\it precision} \color{mybrown}f \color{mypurple}' }} \vspace{5mm} \\
            \vspace{5mm}
            		\LARGE
			\begin{itemize}
				\item {\it Data Type} of the value \vspace{40mm} \\
				\item {\it Field Width} \vspace{40mm} \\
				\item {\it Precision}\vspace{40mm} \\
			\end{itemize}
	
		
		\newpage
		\item \textbf{ \Large escape sequences and the \LARGE{\it fprintf()} \Large function} \vspace{10mm} \\	
			\begin{itemize}
				\item\hspace{15mm} \color{mypurple}\scalebox{1.5}{{\fontfamily{qcr}\selectfont \textbackslash a}} \color{black}\\\\	
				\item\hspace{15mm} \color{mypurple}\scalebox{1.5}{{\fontfamily{qcr}\selectfont \textbackslash b}} \color{black}\\\\	
				\item\hspace{15mm} \color{mypurple}\scalebox{1.5}{{\fontfamily{qcr}\selectfont \textbackslash n}} \color{black}\\\\	
				\item\hspace{15mm} \color{mypurple}\scalebox{1.5}{{\fontfamily{qcr}\selectfont \textbackslash r}} \color{black}\\\\	
				\item\hspace{15mm} \color{mypurple}\scalebox{1.5}{{\fontfamily{qcr}\selectfont \textbackslash t}} \color{black}\\\\	
				\item\hspace{15mm} \color{mypurple}\scalebox{1.5}{{\fontfamily{qcr}\selectfont \textbackslash \textbackslash}} \color{black}\\\\	
				\item\hspace{15mm} \color{mypurple}\scalebox{1.5}{{\fontfamily{qcr}\selectfont \textbackslash `}} \color{black}\\\\
				\item\hspace{15mm} \color{mypurple}\scalebox{1.5}{{\fontfamily{qcr}\selectfont \textbackslash ''}} \color{black}\\\\			
			\end{itemize}
			
			\newpage
			
			\item \textbf{ \Large the \LARGE{\it input()} \Large function} 
            			\Large
            				\begin{itemize}
            					\item adds a simple user interface to your program \\
            					\item you type {\it command window input} \\
            					\item the input can be stored as a variable	\\
            					\item {\bf example:} \\\\
            						The line below goes in your code,\\\\
            						\scalebox{1.5}{{\fontfamily{qcr}\selectfont  x=input(\color{mypurple}`Please type A number '\color{black}) }} \\
            						
            						and then the text will appear in the command window. \\\\
            						\scalebox{1.5}{{\fontfamily{qcr}\selectfont  >>Please type A number\color{black} }} \\
            						
            						Now the user  can type a value (shown in red below) followed by the {\it Enter} key. \\\\	
            						
            						\scalebox{1.5}{{\fontfamily{qcr}\selectfont  >>Please type A number\color{red} \hspace{2MM}42.7}} \\\\
            						
            						Now you can see the value 42.7 is stored in the variable x.
            				\end{itemize}
            				
            			

		\end{itemize}



\end{itemize}


	

\end{document}



