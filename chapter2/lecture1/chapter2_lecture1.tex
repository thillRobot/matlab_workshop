% Lecture Template for ENGR 1120 030 031 E01- Tristan Hill - Spring 2017 - Spring 2018
% 
% Introduction to MATLAB 
%
% THe Command WIndow, more IO
% input and fprintf

% Document settings
\documentclass[11pt]{article}
\usepackage[margin=1in]{geometry}
\usepackage[pdftex]{graphicx}
\usepackage{multirow}
\usepackage{setspace}
\usepackage{hyperref}
\usepackage{color,soul}
\usepackage{fancyvrb}
\usepackage{framed}
\usepackage{wasysym}
\usepackage{multicol}

\pagestyle{plain}
\setlength\parindent{0pt}
\hypersetup{
    bookmarks=true,         % show bookmarks bar?
    unicode=false,          % non-Latin characters in Acrobat’s bookmarks
    pdftoolbar=true,        % show Acrobat’s toolbar?
    pdfmenubar=true,        % show Acrobat’s menu?
    pdffitwindow=false,     % window fit to page when opened
    pdfstartview={FitH},    % fits the width of the page to the window
    pdftitle={My title},    % title
    pdfauthor={Author},     % author
    pdfsubject={Subject},   % subject of the document
    pdfcreator={Creator},   % creator of the document
    pdfproducer={Producer}, % producer of the document
    pdfkeywords={keyword1} {key2} {key3}, % list of keywords
    pdfnewwindow=true,      % links in new window
    colorlinks=true,       % false: boxed links; true: colored links
    linkcolor=red,          % color of internal links (change box color with linkbordercolor)
    citecolor=green,        % color of links to bibliography
    filecolor=magenta,      % color of file links
    urlcolor=blue           % color of external links
}

% assignment number 
\newcommand{\NUM}{2 } 
\newcommand{\VSpaceSize}{2mm} 
\newcommand{\HSpaceSize}{2mm} 

\definecolor{mygreen}{rgb}{0, .39, 0}

%\definecolor{dred}{#8B0000}
% [153,50,204] - dark orchid
\definecolor{mypurple}{rgb}{0.6,0.1961,0.8}
%[139,69,19] - saddle brown
\definecolor{mybrown}{rgb}{0.5451,0.2706,0.0745}

\definecolor{mygray}{rgb}{.6, .6, .6}

\setulcolor{red} 
\setstcolor{green} 
\sethlcolor{mygray} 

\newcommand{\VA}{\vspace{2mm}}
\newcommand{\VB}{\vspace{5mm}} 
 
\newcommand{\R}{\color{red}}
\newcommand{\B}{\color{blue}}
\newcommand{\K}{\color{black}}
\newcommand{\GR}{\color{mygreen}}
\newcommand{\PR}{\color{mypurple}}

\begin{document}

\textbf{ \LARGE ENGR 1120 Lecture Chapter \NUM - \\\\ Vectors and Matrices} \\

\begin{itemize}

	\item \textbf{ \LARGE MATLAB is the {\B Matrix Laboratory}}\\
		\Large
		\begin{itemize}
			\item \textbf{previously we have been using {\PR scalars} }  \\ \vspace{30mm}
			\item \textbf{the default data type is a double precision {\GR matrix}} \\ \vspace{30mm}
			\item \textbf{a matrix is a container (variable) for storing multiple values under the same name}  \\ \vspace{30mm}
				
			\item \textbf{we will begin with {\B 1 - Dimensional} matrices.\\aka - {\PR array} or {\GR vector}}\\  \vspace{30mm}
				
		
		\end{itemize}
	\large	
	\item \textbf{ \LARGE Elements and Indicies}\\
		
		\begin{tabular}{|lccccc|} \hline
        \textbf{ Name:}\hspace{2 mm} Squares&&&&&\\ \hline\hline
        \end{tabular}
        
        \begin{tabular}{|lccccc|} \hline
        \textbf{ Index:}\hspace{2 mm}&1&2&3&4&5 \\ \hline
        \textbf{ Value:}\hspace{4 mm}&14.0&9.0&16.0&25.0&36\\ \hline
        \end{tabular}\hspace{5mm} Example: 1-D matrix named {\GR squares}
        \vspace{30 mm} \\
        
        
        \large	
	\item \textbf{ \LARGE The Memory Bank}\\
        
        \begin{multicols}{2}			
        \begin{tabular}{ | l | c | r | }
            \hline
             \textbf{Name} & \textbf{Memory Address} & \textbf{Value} \\ \hline\hline\hline
            x & 6  & 1.0 \\ \hline\hline\hline
            y & 7 & 99.5 \\ \hline\hline\hline
            z & 8 & 12.7 \\ \hline\hline\hline
            length & 9 & 0.1 \\ \hline\hline\hline
            width & A & 0.3 \\ \hline\hline\hline
            height & B & 0.1 \\ \hline\hline\hline
            grav & C & 9.8 \\ \hline\hline\hline
            -- & D & -- \\ \hline\hline\hline
            -- & E & -- \\ \hline\hline\hline
            -- & F & -- \\ \hline\hline\hline
            -- & 10 & -- \\ \hline	
        \end{tabular}
	
    \vspace{2mm}
        This Memory Bank contains 7 variables.\\ Each is a scalar.
            
        \begin{tabular}{ | l | c | r | }
            \hline
            \textbf{Name} & \textbf{Memory Address} & \textbf{Value} \\ \hline \hline \hline
            a & 11  & 4.3 \\ \hline\hline\hline
            b & 12 & 9.0 \\ \hline\hline\hline
            squares & 13 & 4.0 \\\hline
            -- & 14 & 9.0 \\ \hline
            -- & 15 & 16.0 \\ \hline
            -- & 16 & 25.0 \\ \hline
            -- & 17 & 36.0 \\ \hline\hline\hline
            sum & 18 & 90.0  \\ \hline\hline\hline
            avg & 19 & 18.0 \\ \hline\hline\hline
            -- & 1A & -- \\ \hline\hline\hline
	 -- & 1B & -- \\ \hline\hline\hline
            -- & 1C & -- \\ \hline
                
        \end{tabular}	 
	
	\vspace{2mm}
        This one contains a 1-D matrix called {\GR squares} and 4 scalars.
        \end{multicols}
		
		\newpage
	\item \textbf{ \LARGE using {\PR 1D matrices} in MATLAB }\\
	
		
		\begin{itemize}
			\item \textbf{ \Large Initialization of an Array } \\  \vspace{30mm}
            			
            		\item \textbf{ \Large Accessing } \\  \vspace{30mm}	
            				
            		\item \textbf{ \Large Assignment } \\  \vspace{30mm}			
            				
			\item \textbf{ \Large Re-Assignment , aka Overwrite} \\  \vspace{30mm}			

		\end{itemize}

		\newpage
	\item \textbf{ \LARGE compared to {\B scalars} in MATLAB  }\\
	
		
		\begin{itemize}
			\item \textbf{ \Large Initilization of an Array } \\  \vspace{30mm}
            			
            		\item \textbf{ \Large Accessing } \\  \vspace{30mm}	
            				
            		\item \textbf{ \Large Assignment } \\  \vspace{30mm}			
            				
			\item \textbf{ \Large Re-Assignment , aka Overwrite} \\  \vspace{30mm}			

		\end{itemize}
		
		\newpage
		\item \textbf{ \LARGE some important vocab  }\\
	
		
		\begin{itemize}
			\item \textbf{ \Large \color{blue} elements \color{black} of an \color{mypurple}Array } \\  \vspace{30mm}
            			
            		\item \textbf{ \Large \color{blue}  index \color{black} of an \color{mypurple} element } \\  \vspace{30mm}
            				
            		\item \textbf{ \Large \color{blue} size \color{black} of an \color{mypurple} array} \\  \vspace{30mm}			
            				
			\item \textbf{ \Large \color{blue} shape \color{black} of an \color{mypurple} array} \\  \vspace{30mm}		

		\end{itemize}

			\newpage
		\item \textbf{ \LARGE some useful functions for 1D arrays}\\
	
		
		\begin{itemize}
			\item \textbf{ \Large \color{black} length()} \\  \vspace{15mm}
            		\item \textbf{ \Large \color{black} size()} \\  \vspace{15mm}
            		\item \textbf{ \Large \color{black} sum()} \\  \vspace{15mm}
            		\item \textbf{ \Large \color{black} min()} \\  \vspace{15mm}	
            		\item \textbf{ \Large \color{black} max()} \\  \vspace{15mm}
            		\item \textbf{ \Large \color{black} plot()} \\  \vspace{15mm}		
            		\item \textbf{ \Large \color{black} and many more...} \\  \vspace{15mm}	
		\end{itemize}


\end{itemize}


	

\end{document}



