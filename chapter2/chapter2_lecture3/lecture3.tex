% Lecture Template for ENGR 1120 030 031 E01- Tristan Hill - Spring 2017
% 
% Introduction to MATLAB 
%
% Using 1D arrays 
% Using the Plot Function

% Document settings
\documentclass[11pt]{article}
\usepackage[margin=1in]{geometry}
\usepackage[pdftex]{graphicx}
\usepackage{multirow}
\usepackage{setspace}
\usepackage{hyperref}
\usepackage{color,soul}
\usepackage{fancyvrb}
\usepackage{framed}
\usepackage{wasysym}
\usepackage{multicol}

\pagestyle{plain}
\setlength\parindent{0pt}
\hypersetup{
    bookmarks=true,         % show bookmarks bar?
    unicode=false,          % non-Latin characters in Acrobat’s bookmarks
    pdftoolbar=true,        % show Acrobat’s toolbar?
    pdfmenubar=true,        % show Acrobat’s menu?
    pdffitwindow=false,     % window fit to page when opened
    pdfstartview={FitH},    % fits the width of the page to the window
    pdftitle={My title},    % title
    pdfauthor={Author},     % author
    pdfsubject={Subject},   % subject of the document
    pdfcreator={Creator},   % creator of the document
    pdfproducer={Producer}, % producer of the document
    pdfkeywords={keyword1} {key2} {key3}, % list of keywords
    pdfnewwindow=true,      % links in new window
    colorlinks=true,       % false: boxed links; true: colored links
    linkcolor=red,          % color of internal links (change box color with linkbordercolor)
    citecolor=green,        % color of links to bibliography
    filecolor=magenta,      % color of file links
    urlcolor=blue           % color of external links
}

% assignment number 
\newcommand{\NUM}{2 } 
\newcommand{\VSpaceSize}{2mm} 
\newcommand{\HSpaceSize}{2mm} 

\definecolor{mygray}{rgb}{.6, .6, .6}

% [153,50,204] - dark orchid
\definecolor{mypurple}{rgb}{0.6,0.1961,0.8}
%[139,69,19] - saddle brown
\definecolor{mybrown}{rgb}{0.5451,0.2706,0.0745}

\setulcolor{red} 
\setstcolor{green} 
\sethlcolor{mygray} 

\newcommand{\VA}{\vspace{2mm}}
\newcommand{\VB}{\vspace{5mm}} 
 
\newcommand{\R}{\color{red}}
\newcommand{\B}{\color{blue}}
\newcommand{\K}{\color{black}}
\newcommand{\GR}{\color{mygreen}}
\newcommand{\PR}{\color{mypurple}}


\begin{document}

\textbf{ \LARGE ENGR 1120 Lecture Chapter \NUM - \\\\ More about 1D Matrices} \\

\begin{itemize}

		\item \textbf{ \LARGE We need to remember to clear the workspace ...}\\\\
\begin{itemize}

		 	 
		 \item \textbf{ \LARGE Consider the follow scenario...}\\\\
		\scalebox{1.75}{{\fontfamily{qcr}\selectfont  >> clear {\PR variables}}} \\\\
		
		\scalebox{1.75}{{\fontfamily{qcr}\selectfont  >> X=[5 10 15 20 25 30 35 40 45 50]}} \\\\

		 \item \textbf{ \LARGE If I use `X' again without clearing the data... }\\\\
		 
		 \scalebox{1.75}{{\fontfamily{qcr}\selectfont  >> X=[3 6 9 12 15]}} \\\\
		 
		 \item \textbf{ \LARGE What do you think will happen?}\\\\
\end{itemize}
\newpage
		\item \textbf{ \LARGE Other weird things can happen as well...}\\\\
\begin{itemize}

		 	 
		 \item \textbf{ \LARGE Consider this scenario...}\\\\
		\scalebox{1.75}{{\fontfamily{qcr}\selectfont  >> clear {\PR variables}}} \\\\
		
		\scalebox{1.75}{{\fontfamily{qcr}\selectfont  >> X=[2 4 6 8]}} \\\\
		 
		 \scalebox{1.75}{{\fontfamily{qcr}\selectfont  >> X(6)=99}} \\\\
		 
		 \item \textbf{ \LARGE What do you think will happen?}\\\\
\end{itemize}
\newpage
		\item \textbf{ \LARGE the \color{mypurple}Colon Operator :\color{black}}\\
	
		\Large
		\begin{itemize}
			\item \textbf{ used with sequential arrays that are {\it ranges}} \\\\ i.e. [1, 2, 3, 4, 5, 6, 7, 8, 9, 10] or [0.1, 0.2, 0.3, 0.4, 0.5] \vspace{25mm} 
			
			\item \textbf{ \Large used for Initialization of a range} \\  \vspace{25mm}
			
			\item \textbf{ \Large Slicing} \\  \vspace{15mm}
		\end{itemize}

\newpage
		\item \textbf{ \LARGE Array  \color{mypurple}Concatenation \color{black} }\\
	
		\Large
		\begin{itemize}
			
			\item \textbf{ \Large this is a special use of the \color{mypurple}[ ] \color{black} and {\it initialization}} \\  \vspace{25mm}
			
			\item \textbf{ \Large create arrays from other arrays} \\  \vspace{15mm}
		\end{itemize}

\newpage		
\item \textbf{ \LARGE {\PR Scalar} Operations:}\\
	
	All of the math we have done so far has been {\it Scalar Arithmetic}. This means that each operand was a Scalar (1x1) and each numerical expression was evaluated as a Scalar (1x1). \\
	
	 \scalebox{1.3}{{\fontfamily{qcr}\selectfont  x=10*2\hspace{20mm}\color{blue}--> 20}} \\\\
	  \scalebox{1.3}{{\fontfamily{qcr}\selectfont  y=x*5\hspace{20mm}\color{blue}--> 100}} \\
	  
	  \scalebox{1.3}{{\fontfamily{qcr}\selectfont  A=[10, 15, 12, 13]}} \\\\
	  \scalebox{1.3}{{\fontfamily{qcr}\selectfont  p=A(1)*A(3)\hspace{20mm}\color{blue}--> 120}} \\
		
	
	\item \textbf{ \LARGE {\PR Element-Wise} Operations:}\\
	
	It is often useful to operate on an entire array at once. These operate on the array operands one element at time and generate a array that is the same size and shape as the array operand.
	\begin{itemize}
		\item Element Wise Multiply \scalebox{1.8}{{\fontfamily{qcr}\selectfont  .*}} \\
		\item Element Wise Divide \scalebox{1.8}{{\fontfamily{qcr}\selectfont  ./}} \\
		\item Element Wise Power \scalebox{1.8}{{\fontfamily{qcr}\selectfont  .\^{}}} \\
		
			\scalebox{1.3}{{\fontfamily{qcr}\selectfont  A=[10, 15, 12, 13]}} \\\\
			\scalebox{1.3}{{\fontfamily{qcr}\selectfont  B=[1, 5, 2, 3]}} \\\\
			\scalebox{1.3}{{\fontfamily{qcr}\selectfont  C=A.*B\hspace{20mm}\color{blue}--> 10,    75,    24,    39}} \\\\
			\scalebox{1.3}{{\fontfamily{qcr}\selectfont  C=A./B\hspace{20mm}\color{blue}-->10.0,    3.0,    6.0,  4.3}} \\\\
			\scalebox{1.3}{{\fontfamily{qcr}\selectfont  C=A.\^{}B\hspace{20mm}\color{blue}--> 10, 759375, 144      , 2197}} 
	\end{itemize}
		
\end{itemize}


	

\end{document}



