% Lecture Template for ENGR 1120 020 021- Tristan Hill - Spring 2017 - Summer 2017 - Fall 2017
% 
% Introduction to MATLAB 

% Document settings
\documentclass[11pt]{article}
\usepackage[margin=1in]{geometry}
\usepackage[pdftex]{graphicx}
\usepackage{multirow}
\usepackage{setspace}
\usepackage{hyperref}
\usepackage{color,soul}
\usepackage{fancyvrb}
\usepackage{framed}
\usepackage{wasysym}
\usepackage{multicol}

\pagestyle{plain}
\setlength\parindent{0pt}
\hypersetup{
    bookmarks=true,         % show bookmarks bar?
    unicode=false,          % non-Latin characters in Acrobat’s bookmarks
    pdftoolbar=true,        % show Acrobat’s toolbar?
    pdfmenubar=true,        % show Acrobat’s menu?
    pdffitwindow=false,     % window fit to page when opened
    pdfstartview={FitH},    % fits the width of the page to the window
    pdftitle={My title},    % title
    pdfauthor={Author},     % author
    pdfsubject={Subject},   % subject of the document
    pdfcreator={Creator},   % creator of the document
    pdfproducer={Producer}, % producer of the document
    pdfkeywords={keyword1} {key2} {key3}, % list of keywords
    pdfnewwindow=true,      % links in new window
    colorlinks=true,       % false: boxed links; true: colored links
    linkcolor=red,          % color of internal links (change box color with linkbordercolor)
    citecolor=green,        % color of links to bibliography
    filecolor=magenta,      % color of file links
    urlcolor=blue           % color of external links
}

% assignment number 
\newcommand{\NUM}{1 } 
\newcommand{\VSpaceSize}{2mm} 
\newcommand{\HSpaceSize}{2mm} 

\definecolor{mygray}{rgb}{.6, .6, .6}

\setulcolor{red} 
\setstcolor{green} 
\sethlcolor{mygray} 

\begin{document}

\textbf{ \LARGE ENGR 1120 Lecture -  Chapter \NUM - Introduction} \\\\
\textbf{ \hspace*{5mm}\underline{Matlab: A Practical Introduction to Programming and Problem Solving}\vspace{1mm}\\ 
                \hspace*{5mm}4th ed. by Stormy Attaway }\vspace{2mm}\\
\textbf{ \hspace*{5mm}Tristan Hill - Tennessee Technological University - Spring 2020 } \vspace{3mm}\\


\begin{itemize}


	\item \textbf{ \LARGE 1.1 -  Getting Into MATLAB } (Read this on your own)
	\item \textbf{ \LARGE 1.2 -  The MATLAB environment  } (a.k.a. the I.D.E.)
	
		
		\begin{itemize}
			\item \textbf{\Large The Command Window} \vspace{100mm}
			\item \textbf{\Large The Current Folder} \vspace{100mm}
			\item \textbf{\Large The Workspace} \vspace{100mm}		
			\item \textbf{\Large The Status Indicator} \vspace{50mm}
			\item \textbf{\Large The Layout Menu} \vspace{50mm}		
		\end{itemize}






	\newpage
	\item \textbf{ \LARGE 1.3 - Variables and Assignment Statements} \\
		\begin{itemize}
			\item \Large{A variable is a container for storing values in the RAM. The value that is stored has a {\it type} (1.3.3). We will begin with {\it floating point} values.} \vspace{50mm}
			
		\scalebox{2}{$variable = expression$}	\vspace{50mm}

		
		\newpage
			\item  \Large{There are important rules for choosing a variable name (pg. 9)} \\
				\begin{itemize}
					
					\item The first character  \vspace{35mm}

					\item The rest of the characters  \vspace{35mm}

					\item case matters \vspace{35mm}
					
					\item keywords \vspace{35mm}

					\item function names
				

				\end{itemize}


		\end{itemize}
	\newpage
	\item \textbf{ \LARGE 1.4 - Numerical Expressions}
		\begin{itemize}
			\item \Large{We will begin with numerical expressions. This is how we do typical math computations in MATLAB.} \vspace{10mm}

			\item To do this we need to learn about {\it operator precedence}. \\

				%\begin{multicols}{2}				

				\scalebox{2}{$(\hspace{3mm} )$} \hspace{15mm}\Large{Parenthesis} \\\\

				\scalebox{2}{$\wedge$} \hspace{15mm}\Large{Exponents} \\\\

				\scalebox{2}{$-$} \hspace{15mm}\Large{Negation} \\\\

				\scalebox{2}{$*\hspace{3mm} /$} \hspace{15mm}\Large{Multiplication and Division}\\\\				
	
				 \scalebox{2}{$+\hspace{3mm} -$} \hspace{15mm}\Large{Addition and Subtraction}\\\\
				
				%\end{multicols}	
		
				
		\end{itemize}

\end{itemize}


	

\end{document}



