% Lecture Template for ENGR 1120 030 031 E01- Tristan Hill - Spring 2017
% 
% Introduction to MATLAB 
%
% Writing Your First Program - Hello World - Using Scripts

% Document settings
\documentclass[11pt]{article}
\usepackage[margin=1in]{geometry}
\usepackage[pdftex]{graphicx}
\usepackage{multirow}
\usepackage{setspace}
\usepackage{hyperref}
\usepackage{color,soul}
\usepackage{fancyvrb}
\usepackage{framed}
\usepackage{wasysym}
\usepackage{multicol}

\pagestyle{plain}
\setlength\parindent{0pt}
\hypersetup{
    bookmarks=true,         % show bookmarks bar?
    unicode=false,          % non-Latin characters in Acrobat’s bookmarks
    pdftoolbar=true,        % show Acrobat’s toolbar?
    pdfmenubar=true,        % show Acrobat’s menu?
    pdffitwindow=false,     % window fit to page when opened
    pdfstartview={FitH},    % fits the width of the page to the window
    pdftitle={My title},    % title
    pdfauthor={Author},     % author
    pdfsubject={Subject},   % subject of the document
    pdfcreator={Creator},   % creator of the document
    pdfproducer={Producer}, % producer of the document
    pdfkeywords={keyword1} {key2} {key3}, % list of keywords
    pdfnewwindow=true,      % links in new window
    colorlinks=true,       % false: boxed links; true: colored links
    linkcolor=red,          % color of internal links (change box color with linkbordercolor)
    citecolor=green,        % color of links to bibliography
    filecolor=magenta,      % color of file links
    urlcolor=blue           % color of external links
}

\definecolor{mygreen}{rgb}{0, .39, 0}

%\definecolor{dred}{#8B0000}
% [153,50,204] - dark orchid
\definecolor{mypurple}{rgb}{0.6,0.1961,0.8}
%[139,69,19] - saddle brown
\definecolor{mybrown}{rgb}{0.5451,0.2706,0.0745}

\definecolor{mygray}{rgb}{.6, .6, .6}

\setulcolor{red} 
\setstcolor{green} 
\sethlcolor{mygray} 

\newcommand{\VA}{\vspace{2mm}}
\newcommand{\VB}{\vspace{5mm}} 
 
\newcommand{\R}{\color{red}}
\newcommand{\B}{\color{blue}}
\newcommand{\K}{\color{black}}
\newcommand{\G}{\color{mygreen}}
\newcommand{\PR}{\color{mypurple}}

% assignment number 
\newcommand{\NUM}{1 }
 
\newcommand{\VSpaceSize}{2mm} 
\newcommand{\HSpaceSize}{2mm} 

\begin{document}

\textbf{ \LARGE ENGR 1120 Lecture Chapter \NUM - Hello World \\\\	  Using Scripts and Writing Your First Program} \\
\Large

Previously we have been using MATLAB to solve simple problems by running single commands {\it one-by-one}. You may have noticed that it is inconvient to {\it repeat} this process. \\

\begin{itemize}

	\item \textbf{ \LARGE What is a Program?}\\
		\begin{itemize}
			\item This word has several defintions. \\\\

			\item  In MATLAB we will refer to a program as a {\it script}	
		\end{itemize}

\newpage
	\Large
	\item \textbf{ \LARGE You need to setup and manage a directory for this class!}\\\


\newpage
	\Large
	\item \textbf{ \LARGE Writing Your First Program - Hello World}\\\
		\Large
		\begin{enumerate}
			\item Open the {\bf MATLAB} application.
			
			\newpage
			\item In the {\it Editor} window. Click on the {\bf new } Button. Go down to {\bf script}.	
			
			\newpage
			\item Write a proper {\bf header} at the top of your script. Make sure to include your {\it Name}, the {\it Date}, the {\it Course}, and a {\it Description} of this program.
			
			\newpage
			\item In the {\it Editor} window. Click on the {\bf save} button. Now you will need to name your file and save it in your directory structure.
			
			\newpage
			\item Now you are going to start {\bf writing} your first program. 
			
			\newpage
			\item {\bf Run} your program and {\it watch the magic}!
		\end{enumerate}

\end{itemize}


	

\end{document}



